% Author: Nelson Lago
% This file is distributed under the MIT Licence

%%%%%%%%%%%%%%%%%%%%%%%%%%%%%%%%%%%%%%%%%%%%%%%%%%%%%%%%%%%%%%%%%%%%%%%%%%%%%%%%
%%%%%%%%%%%%%%%%%%%%%%%%%%%%%%%%% PREÂMBULO %%%%%%%%%%%%%%%%%%%%%%%%%%%%%%%%%%%%
%%%%%%%%%%%%%%%%%%%%%%%%%%%%%%%%%%%%%%%%%%%%%%%%%%%%%%%%%%%%%%%%%%%%%%%%%%%%%%%%

% A língua padrão é a última da lista
\documentclass[a1paper,brazilian,english]{article}

% Vários pacotes e opções de configuração genéricos
\usepackage{imegoodies}
\usepackage[poster,hidelinks]{imelooks}
% \tcbposterset{fontsize = 32pt} % default, mude se necessário

% Diretórios onde estão as figuras; com isso, não é necessário (mas
% é permitido) colocar o caminho completo em \includegraphics. Note
% que a extensão nunca é necessária (mas é permitida), ou seja, o
% resultado é o mesmo com "\includegraphics{figuras/foto.jpeg}",
% "\includegraphics{foto.jpeg}", "\includegraphics{figuras/foto}"
% ou "\includegraphics{foto}".
\graphicspath{{figuras/},{fig/},{logos/},{img/},{images/},{imagens/}}

% Comandos rápidos para mudar de língua:
% \en -> muda para o inglês
% \br -> muda para o português
% \texten{blah} -> o texto "blah" é em inglês
% \textbr{blah} -> o texto "blah" é em português
\babeltags{br = brazilian, en = english}


%%%%%%%%%%%%%%%%%%%%%%%%%%%%%%%%%%%%%%%%%%%%%%%%%%%%%%%%%%%%%%%%%%%%%%%%%%%%%%%%
%%%%%%%%%%%%%%%%%%%%%%%%%%%%%%%%%% METADADOS %%%%%%%%%%%%%%%%%%%%%%%%%%%%%%%%%%%
%%%%%%%%%%%%%%%%%%%%%%%%%%%%%%%%%%%%%%%%%%%%%%%%%%%%%%%%%%%%%%%%%%%%%%%%%%%%%%%%

% O arquivo com os dados bibliográficos para biblatex; você pode usar
% este comando mais de uma vez para acrescentar múltiplos arquivos
\addbibresource{bibliografia.bib}

% Este comando permite acrescentar itens à lista de referências sem incluir
% uma referência de fato no texto (pode ser usado em qualquer lugar do texto)
%\nocite{bronevetsky02,schmidt03:MSc, FSF:GNU-GPL, CORBA:spec, MenaChalco08}
% Com este comando, todos os itens do arquivo .bib são incluídos na lista
% de referências
%\nocite{*}


%%%%%%%%%%%%%%%%%%%%%%%%%%%%%%%%%%%%%%%%%%%%%%%%%%%%%%%%%%%%%%%%%%%%%%%%%%%%%%%%
%%%%%%%%%%%%%%%%%%%%%%%%%%%%%%% INÍCIO DO POSTER %%%%%%%%%%%%%%%%%%%%%%%%%%%%%%%
%%%%%%%%%%%%%%%%%%%%%%%%%%%%%%%%%%%%%%%%%%%%%%%%%%%%%%%%%%%%%%%%%%%%%%%%%%%%%%%%


% Existem várias packages para criar pôsteres com LaTeX (a0poster, baposter,
% tikzposter, sciposter...). As mais comuns atualmente são beamerposter
% e tcolorbox (com sua biblioteca "poster"). Ambas funcionam muito bem;
% beamerposter é mais familiar (ela simplesmente utiliza beamer com alguns
% ajustes no tamanho das fontes e do papel), mas com tcolorbox o alinhamento
% vertical dos elementos é MUITO mais simples, e esta é a solução adotada
% aqui. Vale muito a pena ler a documentação com "texdoc tcolorbox" e
% "texdoc tcolorbox-tutorial-poster".

% Um pôster com tcolorbox é composto por blocos (posterboxes) coloridos
% de tamanho variável; cada bloco pode conter textos ou imagens e um
% título opcional. O pôster utiliza uma grade de dimensões definidas em
% \begin{tcposter} com "rows=" e "columns=" para fazer o alinhamento:
% para cada posterbox, podemos dizer "row=X, column=Y" para definir sua
% posição. Além disso, podemos dizer "span=A, rowspan=B" para fixar
% seu tamanho. Sem "span" e "rowspan", uma posterbox tem pelo menos o
% tamanho de uma célula da grade, mas se seu tamanho natural for maior
% ela extrapola esse tamanho. "span" e "rowspan" podem ser números
% não-inteiros (como 0.8 ou 1.4).
%
% "\begin{posterbox}" recebe um conjunto de parâmetros opcional e um
% conjunto de parâmetros obrigatório:
%
% "\begin{posterbox}[opcional]{obrigatório}".
%
% O conjunto de parâmetros opcional é onde inserimos os parâmetros comuns
% de tcolorbox, como "adjusted title", "coltext", "titlerule" etc.; o
% conjunto de parâmetros obrigatório é usado para determinar as dimensões
% e a posição da posterbox, ou seja, as opções "name", "column", "below",
% "span" etc.
%
% ALINHAMENTO HORIZONTAL
%
% É possível definir um poster com 2 colunas e fazer algo como
%
% \posterbox{column=1, span=1.3}{blah}
% \posterbox{column*=2, span=0.7}{blah}
%
% A segunda posterbox será alinhada à direita ("column*="), então as
% duas serão colocadas lado-a-lado sem sobreposições.
%
% Na prática, no entanto, é mais fácil fazer como no exemplo abaixo:
% definimos que o poster tem 12 colunas, o que nos permite dividir
% sua largura em 2, 3, 4 ou 6 colunas iguais ou diferentes (como
% 1/2 + 1/2, 2/3 + 1/3, 1/4 + 1/4 + 1/2, 1/4 + 1/6 + 1/4 + 1/3 etc).
%
% ALINHAMENTO VERTICAL
%
% Embora seja possível alinhar as posterboxes em função da grade na
% vertical, uma outra possibilidade é utilizar "above", "below" e
% "between", como no exemplo abaixo: basta associar um nome "blah" a
% uma determinada posterbox e, em outra, dizer "below=blah". Lembre-se
% que a posterbox de nome "blah" deve ser definida *antes* que outra
% possa fazer referência a ela. Também é possível fazer "below=top",
% "above=bottom" etc. A opção "equal height group" também é muito útil.
% Nada impede que você use estratégias de alinhamento diferentes para
% cada posterbox.

% Este modelo define a opção "smallmargins", que diminui a distância
% entre o conteúdo de uma posterbox e suas bordas. Use com parcimônia!

\begin{document}

% Em um poster não há \maketitle

\begin{tcbposter}[
  poster = {
    %showframe, % muito útil durante a preparação do poster
    rows = 6,
    columns = 12,
    colspacing = 1.2cm,
    rowspacing = .8cm,
  },
]


\posterbox[titlebox]{name=titlebox, below=top, column=1, span=12}{
    Automatização do fluxo de submissões de patches para o kernel Linux através do kworkflow\\
    {\small João Guilherme Barbosa de Souza}
}

\posterbox[footerbox]{name=footerbox, above=bottom, column=1, span=12}{
    \begin{minipage}[t]{0.55\textwidth}
        \large
        <joaosouzaaa12@usp.br>\par
        \vspace{4pt}
        \footnotesize\rmfamily
        \textcolor{imesoftblue!30!white}
          {Instituto de Matemática, Estatística e Ciência da Computação}\relax\par
        \textcolor{imesoftblue!30!white}
          {Universidade de São Paulo}\relax
    \end{minipage}%
    \hfill
    \begin{minipage}[t]{0.45\textwidth}
        \raggedleft
        \large
        \textbf{Orientadores}\par
        \vspace{4pt}
        \footnotesize\rmfamily
        \textcolor{imesoftblue!30!white}
          {David Tadokoro}\relax\par
        \textcolor{imesoftblue!30!white}
          {Paulo Meirelles}\relax
    \end{minipage}
}

% \posterbox[footerbox]{name=footerbox, above=bottom, column=1, span=12}{
%     \large
%     João Guilherme Barbosa de Souza\par
%     \vspace{4pt}
%     \small\ttfamily
%     joaosouzaaa12@usp.br\par
%     \vspace{4pt}
%     \footnotesize\rmfamily
%     \textcolor{imesoftblue!30!white}
%       {Instituto de Matemática, Estatística e Ciência da Computação}\relax
% }

%%%%%%%% Contexto: Desenvolvimento kernel Linux %%%%%%%%


\posterbox[adjusted title = Contribuições no kernel Linux, smallmargins, equal height group = toprow, colback=white, before upper=\justifying]
          {name=contexto, below=titlebox, column=1, span=6}{

    \noindent O modelo de desenvolvimento do kernel Linux é colaborativo, descentralizado e contínuo. Submeter, acompanhar e enviar novas versões de \textit{patches} ocorrem via listas de e-mail. Por conta disto, nota-se que um tempo considerável é gasto com processos manuais e/ou configurações de ferramentas externas.\\

    \begin{center}
        \includegraphics[width=0.72\textwidth]{figuras/patch-lifecycle.png}
        \captionof{figure}{Caminho de uma contribuição no kernel Linux}
        \label{fig:patch-lifecycle}
    \end{center}
}

%%%%%%%% Kworkflow %%%%%%%%

\posterbox[adjusted title = Kworkflow, smallmargins, equal height group = toprow, colback=white, before upper=\justifying]
          {name=kworkflow, below=titlebox, column=7, span=6}{

    \noindent Kworkflow (\textbf{kw}) é um hub de ferramentas para automação de tarefas comuns no desenvolvimento do kernel Linux, composto de soluções \textit{in-house} e integrações. A fim de preencher lacunas no processo de contribuição não cobertas pelo projeto, propomos duas novas features:\\ \textbf{kw manage-contacts} e \textbf{kw patch-track}.

    \begin{center}
        \includegraphics[width=0.7\textwidth]{figuras/kw-architecture.pdf}
        \captionof{figure}{Arquitetura do kw}
        \label{fig:kw-architecture}
    \end{center}
}

%%%%%%%% kw manage-contacts funcionalidades %%%%%%%%

\posterbox[adjusted title = Fechando o fluxo de submissão de \textit{patches} no \textbf{kw}, smallmargins, equal height group = midrow, colback=white, before upper=\justifying]
          {name=fechando-fluxo-submissoes, below=contexto, column=1, span=12} {
    \begin{minipage}{0.46\textwidth}
        \centering
        \includegraphics[width=\linewidth]{figuras/diagrama-tcc-atuacao-implementacoes.png}
        \captionof{figure}{Fluxo completo com auxílio do kw}
        \label{fig:submission-flow}
    \end{minipage}
    \hfill
    \begin{minipage}{0.49\textwidth}

        \noindent Funcionalidades do \textbf{kw manage-contacts}:
        \begin{itemize}
            \item Criar, renomear, remover e adicionar contatos à grupos
            \item Visualizar grupos e contatos associados ao grupo
            \item Realizar submissões diretamente à grupos via ferramenta \textbf{kw send-patch}
        \end{itemize}

        \hfill

        \noindent Funcionalidades do \textbf{kw patch-track}:
        \begin{itemize}
            \item Registro automático de submissões via \textbf{kw send-patch}
            \item Atualização manual do estado dos patches ou via heurística com comando \textbf{kw patch-track update}
            \item Visualizar lista de submissões ou lista de patches por submissão
            \item Visualização dos e-mails submetidos e respostas via \textbf{mutt}
        \end{itemize}
    \end{minipage}
}

%%%%%%%% kw patch-track funcionalidades %%%%%%%%

% \posterbox[adjusted title = Funcionalidades do kw patch-track, smallmargins, equal height group = midrow, colback=white, before upper=\justifying]
%           {name=patch-track, below=contexto, column=7, span=6} {
%     \begin{itemize}
%     \item Registro automático de submissões via \textbf{kw send-patch}
%     \item Atualização manual do estado dos patches ou via heurística com comando \textbf{kw patch-track update}
%     \item Visualizar lista de submissões ou lista de patches por submissão
%     \item Visualização dos e-mails submetidos e respostas via \textbf{mutt}
%   \end{itemize}
%
% }

%%%%%%%% Resultados %%%%%%%%

\posterbox[adjusted title = Resultados, smallmargins, equal height group = botrow, colback=white, before upper=\justifying]
          {name=resultados, below=fechando-fluxo-submissoes, column=1, span=6} {
  \begin{itemize}
    \item Melhora submissão e corretude com uso de grupos de email
    \item Visibilidade do estado e evolução dos patches diretamente no kw
    \item Reduz tempo de consulta do histórico e da mailing list
    \item Reduz perda de contexto com registro local das revisões
    \item Minimiza alternância entre ferramentas externas
  \end{itemize}
}

%%%%%%%% Próximos passos %%%%%%%%

\posterbox[adjusted title = Próximos passos, smallmargins, equal height group = botrow, colback=white, before upper=\justifying]
          {name=proximos-passos, below=fechando-fluxo-submissoes, column=7, span=6} {
  \begin{itemize}
    \item Melhorar gestão de submissões/contribuições
    \item Melhorar heurística de atualização do estado da contribuição
    \item Desenvolvimento e integração com módulo de revisão local
    \item Uso de Mailboxes específicas
  \end{itemize}
}

\end{tcbposter}

\end{document}
