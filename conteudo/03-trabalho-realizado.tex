\chapter{Contribuições}

\todo[inline]{Será que fica legal um título mais ousado? Algo mais relacionado com a lacuna que você está preenchendo nos workflows do kernel e o "fechamento" da trilogia rubens, eu e você.}

Esse trabalho dá continuidade à um processo de melhoria continua ao software do kw, iniciada anteriormente
através de outros trabalhos como o \textit{Simplificando o processo de contribuição para o kernel Linux} de \cite{gomes2022kernelworkflow},
que estrutura e refatora a documentação da ferramenta, implementa a versão inicial do banco de dados e também da funcionalidade \textit{kw mail}, posteriormente renomeada para \textit{kw send\_patch}, utilizada para 
submissão de patches através do envio de email's; e também o trabalho \textit{Integrating the Kworkflow system with the Lore archives: Enhancing the Linux kernel developer interaction with mailing lists},
desenvolvido por \cite{tadokoro2023kwlore}, que implementa o \textit{patch-hub} -- interface de terminal para os arquivos Lore, permitindo o acesso à uma lista oficial de discussões e patches do Kernel Linux.

Como proposta, as contribuições oferecidas por esse trabalho focam em oferecer melhorarias e automatizações para a gestão de patches submetidos por parte dos contribuidores do kernel linux enquanto estão sob processo de revisão, integrando-se ao fluxo de implementações de seus predecessores que, respectivamente, introduzem o processo de envio e de consulta de patches já enviados.

\section{CRUD banco de dados}

Para poder dar suporte para suas diversas features, o kw conta com um sistema de banco de dados, desenvolvido em SQLite3, que armazena informações necessárias para o funcionamento, principalmente, das as ferramentas kw pomodoro e kw patch-hub, além de possuir dados de telemetria sobre a utilização do software pelos usuários. Para garantir a consistência e segurança dos dados entre o banco de dados e a aplicação, é crucial desenvolver operações que lidem com operações de manipulação, como inserção, leitura, atualização e deleção de dados (conhecidas como CRUD - create, read, update, delete). Essas operações servem como interface entre as diferentes partes do sistema, permitindo uma interação eficaz e garantindo que os dados sejam gerenciados de forma precisa e confiável.

Contudo, deixar instruções SQL dispersas diretamente no código de aplicação não é considerado uma boa prática de engenharia de software, pois dificulta a manutenção, a legibilidade e a evolução do sistema. O ideal é encapsular o acesso ao banco de dados em funções ou camadas de abstração que forneçam operações de mais alto nível, reduzindo o acoplamento entre a lógica de negócio e as consultas.

No caso do Kworkflow, desenvolvido em Bash, tal abordagem é limitada pela própria linguagem, que não dispõe de mecanismos nativos para abstração de consultas SQL. Assim, a interação com o banco de dados precisa ser realizada diretamente por meio de comandos de script, o que torna essa separação menos natural, embora ainda desejável para organizar e isolar responsabilidades.

Para tal, o software contava com algumas funções implementadas que permitiam a interação com o banco de dados, mas que não isolavam suficientemente o código e as queries, tornando necessário o uso de comandos SQL em alguns casos. Esse cenário era visível nos comandos de seleção originais (Programa~\ref{prog:select_from_antigo}), que recebiam trechos na linguagem SQL com a cláusula \textit{where}, utilizada para especificar quais critérios devem atender os parâmetros selecionados ou removidos, como exemplificado no Programa~\ref{prog:snippet_uso_antigo_select}.

\begin{programruledcaption}{código select\_from antigo. \label{prog:select_from_antigo}}
    \inputminted[breaklines,fontsize=\footnotesize]{bash}{conteudo/implementacoes/codigos/db/select_from_antigo.bash}
\end{programruledcaption}

\begin{programruledcaption}{snippet uso função select\_from\_antiga. \label{prog:snippet_uso_antigo_select}}
    \inputminted[breaklines,fontsize=\footnotesize]{bash}{conteudo/implementacoes/codigos/db/snippet_uso_antigo_select_from.bash}
\end{programruledcaption}

Além disso, especificamente nos comandos de seleção (Programa~\ref{prog:select_from_novo}), também foi implementado um parâmetro adicional para permitir o uso da clausula \textit{ordered\_by}, que possibilita especificar uma ordenação para os dados retornados com base em um atributo comparável entre eles. O resultado dessa transição para o uso de parâmetros pode ser observado no Programa~\ref{prog:snippet_uso_novo_select}.

\begin{programruledcaption}{código select\_from novo. \label{prog:select_from_novo}}
    \inputminted[breaklines,fontsize=\footnotesize]{bash}{conteudo/implementacoes/codigos/db/select_from_novo.bash}
    
\end{programruledcaption}

\begin{programruledcaption}{snippet uso função select\_from\_nova. \label{prog:snippet_uso_novo_select}}
    \inputminted[breaklines,fontsize=\footnotesize]{bash}{conteudo/implementacoes/codigos/db/snippet_uso_novo_select_from.bash}
\end{programruledcaption}

Por fim, essa alteração também permitiu que comparações de desigualdades fossem feitas de forma mais abrangente e ordenada, visto que na função de remoção antiga (Programa~\ref{prog:remove_from_antigo}) apenas operações de comparação eram possíveis e que na função de seleção apenas com o código SQL explícito. Para modernizar esse fluxo, implementou-se a nova função de remoção (Programa~\ref{prog:remove_from_novo}) baseada na lógica de geração automática de cláusulas do Programa~\ref{prog:generate_where_clause}.

\begin{programruledcaption}{código remove\_from antigo. \label{prog:remove_from_antigo}}
    \inputminted[breaklines,fontsize=\footnotesize]{bash}{conteudo/implementacoes/codigos/db/remove_from_antigo.bash}
\end{programruledcaption}

\begin{programruledcaption}{código generate\_where\_clause utilizado nas novas funções para gerar a cláusula WHERE SQL a partir dos parâmetros passados. \label{prog:generate_where_clause}} 
    \inputminted[breaklines,fontsize=\footnotesize]{bash}{conteudo/implementacoes/codigos/db/generate_where_clause.bash} 
\end{programruledcaption}

\begin{programruledcaption}{código remove\_from novo. \label{prog:remove_from_novo}. \label{prog:update_into}} 
    \inputminted[breaklines,fontsize=\footnotesize]{bash}{conteudo/implementacoes/codigos/db/remove_from_novo.bash}  
\end{programruledcaption}

Além disso, outra implementação desenvolvida nessa etapa, foi a implementação do novo método \textit{update\_into} (Programa~\ref{prog:update_into}), que permitia a alteração pontual de algum atributo dentro de uma entidade do banco de dados e da função \textit{generate\_set\_clause} (Programa~\ref{prog:generate_set_clause}), utilizada para gerar a clausula \textit{set} do SQL, que define quais conjuntos de atributos serão alterados e quais os novos valores para esses atributos. Essa implementação também faz uso da função \textit{generate\_where\_clause}, uma vez que na maioria das alterações se faz necessário especificar qual entidade/conjunto de entidades receberá as alterações, como exemplificado no Programa~\ref{prog:snippet_update_into}.

\begin{programruledcaption}{código update\_into}
    \inputminted[breaklines,fontsize=\footnotesize]{bash}{conteudo/implementacoes/codigos/db/update_into.bash}
\end{programruledcaption}

\begin{programruledcaption}{código generate\_set\_clause utilizado para permitir especificar quais atributos serão alterados e quais serão seus novos valores. \label{prog:generate_set_clause}} 
    \inputminted[breaklines,fontsize=\footnotesize]{bash}{conteudo/implementacoes/codigos/db/generate_set_clause.bash}
\end{programruledcaption}

\begin{programruledcaption}{snippet uso função update\_into. \label{prog:snippet_update_into}} 
    \inputminted[breaklines,fontsize=\footnotesize]{bash}{conteudo/implementacoes/codigos/db/snippet_update_into_uso.bash}
\end{programruledcaption}

A padronização dessas operações de CRUD e o isolamento das consultas SQL em funções parametrizadas foram fundamentais para garantir a escalabilidade do sistema. Esta base técnica de persistência de dados permitiu o desenvolvimento de funcionalidades que exigem um gerenciamento mais complexo de informações, como a automação de contatos e grupos, que será detalhada na seção seguinte.
\section{KW Manage Contacts}

\todo[inline]{Adicionar ``intro da seção''}

\subsection{Objetivos}

No fluxo atual, patches podem ser submetidos através do kw com o uso da ferramenta \textit{kw send-patch}, que recebe como dados a lista de commits que deverão ser enviados e a lista
de usuários, em geral, mantenedores\todo{faltou as listas do subsistema/contexto em questão}, que devem ser notificados da submissão do patch. A partir desse comando, um patch será gerado com as alterações especificadas e, através do \textit{git email},
ferramenta de emails do \textit{github} os usuários receberão um email com as implementações do usuário.\todo[inline]{Na verdade, o kw usa o git send-email por baixo dos panos e ele não tem relação com o GitHub. Aqui vale pincelar como a noção de hub do kw, em que ele agrega a ferramenta que já é consolidada (de certa forma), facilitando e aprimorando o seu uso.}

Ao lidar com e-mails, é sempre comum que hajam grupos de pessoas que sempre serão endereçadas durante a submissão. 
Atualmente, o kw fornece uma funcionalidade, o \textit{get\_maintainers}\todo{get\_maintainers é o script por baixo dos panos; no kw é só kw maintainers} que lista
mantenedores responsáveis pelos sistemas alterados, facilitando a identificação do ou dos responsáveis,
o que, porém, não atende alguns grupos não oficiais, como, por exemplo, colegas de trabalho ou outros grupos
externos envolvidos com o patch. Dessa forma, havia a necessidade de uma ferramenta que pudesse oferecer um sistema de gerenciamento de grupos
de e-mail integrado ao fluxo do kw, garantindo maior praticidade e consistência na comunicação.

Assim, o objetivo principal foi o de criar um sistema que permitisse gerenciar grupos de e-mail de forma centralizada, com armazenamento persistente em
banco de dados\todo{talvez ``[...] armazenamento persistente no banco de dados do kw [...]''} e acesso através de interface em linha de comando (CLI). Permitindo, através disso, que o usuário pudesse cadastrar contatos, organizar esses contatos em
grupos e, posteriormente, incluir automaticamente tais grupos ao enviar patches utilizando o kw send-patch.

\subsection{Arquitetura}

A arquitetura da solução foi planejada de forma modular. O banco de dados é responsável por armazenar contatos individuais (\textit{email\_contact}), os grupos (\textit{email\_group}) e suas associações (\textit{email\_contact\_group}),
enquanto a interface de linha de comando fornece os comandos necessários para manipulação dessas informações. O modelo de dados contempla as entidades contato,
grupo e a relação entre elas, garantindo flexibilidade para gerenciar múltiplos contextos e equipes.\todo[inline]{Seria legal colocar os trechos do kwdb.sql que você implementou. Lembre de omitir o que não for de interesse, pois podemos colocar na íntegra o volume do que você fez nos anexos.}

\subsection{Funcionalidades}
A interação com a ferramenta ocorre exclusivamente pelo terminal, de forma a manter compatibilidade com o fluxo tradicional do kw. 
Foram definidos comandos claros e diretos, permitindo que o usuário visualize grupos existentes, adicione novos contatos, associe-os a
diferentes grupos e utilize esses grupos diretamente no envio de e-mails.

As principais funcionalidades implementadas incluem:\todo{Ao invés de usar os dois pontos, usar o \textbackslash ref\{\} funciona melhor (``[...] principais funcionalidades implementadas estão ilustradas no Programa X.YZ.''}

\begin{programruledcaption}{comandos kw manage contacts}
    \inputminted[breaklines,fontsize=\footnotesize]{bash}{conteudo/implementacoes/codigos/kw_manage_contacts/comandos.bash}
\end{programruledcaption}

\subsubsection{Criação de Grupos}
Essa função recebe o nome de um novo grupo e realiza a validação antes de sua criação. 
A validação consiste em verificar se o nome já não existe no banco de dados, 
se não contém caracteres especiais e se o tamanho é inferior a 50 caracteres. 
Caso todas as condições sejam atendidas, o grupo é criado e persistido no banco com sucesso.\todo{ref ao programa}

\begin{programruledcaption}{create\_email\_group e create\_group}
    \inputminted[breaklines,fontsize=\footnotesize]{bash}{conteudo/implementacoes/codigos/kw_manage_contacts/create_email_group_e_create_group.bash}
\end{programruledcaption}

\subsubsection{Exclusão de Grupos}
Essa função remove um grupo de e-mails específico e todas as suas referências no banco de dados.
Nesse caso, é verificada apenas a existência do grupo, uma vez que, se já foi previamente adicionado, 
deve atender aos critérios de validação. A remoção utiliza a cláusula CASCADE na tabela de associação 
\textit{email\_contact\_group}\todo{se der, mostrar a modelagem}, garantindo que todas as relações existentes para aquele grupo sejam automaticamente 
excluídas. Além disso, os contatos que permanecerem sem associação a grupos também são removidos.\todo{ref ao programa}

\begin{programruledcaption}{remove\_email\_group e remove\_group}
    \inputminted[breaklines,fontsize=\footnotesize]{bash}{conteudo/implementacoes/codigos/kw_manage_contacts/remove_email_group_e_remove_group.bash}
\end{programruledcaption}

\subsubsection{Renomeação de Grupos}
Essa função realiza a renomeação de um grupo de e-mails existente, 
recebendo como parâmetros o nome atual e o novo nome desejado. Antes da atualização,
o novo nome do grupo passa novamente pelo processo de validação, garantindo que não corresponda 
a um grupo já existente no banco de dados e que siga as mesmas regras de restrição aplicadas na 
criação — como não conter caracteres especiais e ter comprimento inferior a 50 caracteres. 
Se as condições forem atendidas, o nome é atualizado com sucesso.\todo{ref ao programa e, se der, mostrar e ref a parte da modelagem}

\begin{programruledcaption}{rename\_email\_group e rename\_group}
    \inputminted[breaklines,fontsize=\footnotesize]{bash}{conteudo/implementacoes/codigos/kw_manage_contacts/rename_email_group_e_rename_group.bash}
\end{programruledcaption}

\subsubsection{Adicionar contatos à Grupos de Email}
Essa função é responsável por associar novos contatos a um grupo de e-mails existente. 
Ela recebe como parâmetros o nome do grupo e uma lista de contatos no formato ``NOME\_CONTATO <EMAIL\_CONTATO>, NOME\_CONTATO <EMAIL\_CONTATO>,
...''. Inicialmente, a função valida a existência do grupo no banco de dados. Em seguida, divide a lista de contatos e executa verificações
individuais para cada um, como a validade do endereço de e-mail e a presença de um nome associado. Após essas validações, os contatos são 
inseridos no banco de dados e vinculados ao grupo correspondente.\todo{ref ao programa e, se der, mostrar e ref a parte da modelagem}

\begin{programruledcaption}{add\_email\_contacts e add\_contact\_group}
    \inputminted[breaklines,fontsize=\footnotesize]{bash}{conteudo/implementacoes/codigos/kw_manage_contacts/add_email_contacts_e_add_contact_group.bash}
\end{programruledcaption}

\subsubsection{Exibir grupos}
A função show\_email\_groups\todo{ref ao programa e colocar em itálico como você estava convencionando} é utilizada para exibir informações sobre os grupos de e-mail cadastrados. 
Caso receba como parâmetro o nome de um grupo específico, a função valida sua existência e, 
em seguida, mostra os contatos associados a esse grupo. Por outro lado, quando nenhum parâmetro 
é informado, são exibidas informações gerais sobre todos os grupos existentes no banco de dados. 

\begin{programruledcaption}{show email groups, print\_groups\_infos e print\_contacts\_infos}
    \inputminted[breaklines,fontsize=\footnotesize]{bash}{conteudo/implementacoes/codigos/kw_manage_contacts/show_email_groups.bash}
\end{programruledcaption}

\begin{figure}[H]
  \centering
    \includegraphics[width=1.2\textwidth]{exemplo\_group\_show}
    \caption{Exemplo do comando ``group\_show''  sem um grupo especificado.\label{fig:exemplo_group_show:a}}
\end{figure}

\begin{figure}[H]
  \centering
    \includegraphics[width=1.2\textwidth]{exemplo_group_show_especifico}
    \caption{Exemplo do comando ``group\_show'' para um grupo específico.\label{fig:exemplo_group_show_especifico:a}}
\end{figure}

\subsection{Enviar patches para grupos}

Dois novos comandos, \textit{--to-groups} e \textit{--cc-groups} foram adicionados à ferramenta \textit{send-patch} para permitir o envio de patches diretamente para grupos.\todo{ref ao programa}
Esses comandos recebem listas com os nomes dos grupos separados por vírgulas, busca os contatos associados no banco de dados e adiciona seus emails como destinatários.

\begin{programruledcaption}{opções do comando --send do kw send-patch contendo o to-groups e o cc-groups}
    \inputminted[breaklines,fontsize=\footnotesize]{bash}{conteudo/implementacoes/codigos/kw_manage_contacts/kw_send_patch_options.bash}
\end{programruledcaption}

\begin{programruledcaption}{Função send\_patch\_main com métodos --to-groups e cc-groups}
    \inputminted[breaklines,fontsize=\footnotesize]{bash}{conteudo/implementacoes/codigos/kw_manage_contacts/kw_send_patch.bash}
\end{programruledcaption}

\subsection{Resultados}
Entre os benefícios da abordagem adotada estão a maior praticidade no gerenciamento de destinatários, a redução de erros manuais na inclusão
de e-mails e a possibilidade de reutilização de grupos em diferentes contextos. Isso se traduz em um processo mais ágil e confiável no envio de patches.

Apesar dos avanços alcançados, algumas limitações ainda podem ser apontadas. 
A ferramenta oferece suporte apenas via CLI, não possuindo interface gráfica que poderia ser desejável, principalmente para visualizar
informações dos grupos e contatos de uma maneira mais organizada.\todo[inline]{Acho que, ao invés de falar de GUI, falar de formas de se importar ou até exportar os grupos. Sei que existem aqueles arquivos de contact books, mas veja se procede essa ideia; de toda forma, GUI não é muito a linha do kw. Se estiver se referindo a uma TUI na linha do patch-hub aí faz sentido, mas seja explícito, porque interface gráfica é associado a GUI.}

\section{KW Patch track}

Atualmente, embora seja possível submeter patches através do kw, ainda não existe um mecanismo eficaz para acompanhar e gerenciar o ciclo de vida dessas submissões. Conforme novas versões de um mesmo patch são enviadas e revisões se acumulam, tornando cada vez mais difícil manter o controle sobre o histórico, as respostas recebidas e o estado atual de cada revisão.

\subsection{Objetivos}
Diante dessa limitação, surgiu\todo{talvez sugerir que esta lacuna é significativa, pois faz com que o usuário do kw tenha que resolver isto ``por fora''} a necessidade de uma ferramenta capaz de registrar, rastrear e atualizar automaticamente o status dos patches submetidos. Assim, o objetivo principal do \textit{Patch Track} é permitir que o usuário acompanhe de forma automatizada o progresso de suas contribuições, desde o envio inicial até a integração no repositório, reduzindo o esforço manual e promovendo maior clareza sobre o processo de revisão.

Além disso, o sistema busca oferecer uma base sólida para extensões futuras, como integração com repositórios oficiais e coleta de métricas sobre o fluxo de contribuição, incluindo tempo médio de resposta, aprovação e integração de patches.\todo[inline]{Mais uma vez (prometo que a última), bata na tecla do kw como software para pesquisa e como ter dados como estes podem ser interessantes para estudar o modelo de desenvolvimento do kernel Linux.}

\todo[inline]{Acho bom deixar claro que o patch-track ainda não está de fato implementado, mas que a parte mais difícil e os componentes gerais já foram feitos (no geral, veja como valorizar o que você fez).}

\subsection{Arquitetura}

A arquitetura do \textit{Patch Track} foi projetada de forma modular e baseada em um modelo relacional de entidades interligadas. Todas as informações são armazenadas em banco de dados, garantindo rastreabilidade e consistência das submissões.

A entidade central, \textit{patch}, armazena informações como o autor, o \textit{message-id} da submissão, a versão e o status atual do patch. O campo \textit{outdated} indica quando uma versão mais recente substitui outra, preservando o histórico completo das alterações.

A entidade \textit{contribution} agrupa logicamente diferentes versões de um mesmo trabalho, mantendo informações sobre a data da última interação e o repositório de destino. Esse repositório é representado pela entidade \textit{repository}, que contém dados como nome, URL e \textit{branch} associada na qual a contribuição deve ser integrada assim que aprovada, além dos mantenedores vinculados à esse repositório, permitindo identificar revisores e correlacionar respostas relevantes nas threads de e-mail.

O rastreamento dos envios é realizado pela tabela \textit{patch\_submission}, que registra o identificador da mensagem, o remetente e o vínculo entre cada envio e o patch correspondente. O sistema também oferece suporte a \textit{tags}, utilizadas como marcadores semânticos para facilitar a filtragem, a categorização e a exibição das informações.

Essa estrutura de dados estabelece uma base robusta para o controle do ciclo de vida dos patches e possibilita futuras expansões, como integração com serviços externos de revisão e automação de métricas analíticas.

\todo[inline]{Veja se resgata aquele pseudo-UML que você me mostrou pra mim naquela reunião em setembro quando eu estava no CBSoft. Isto valoriza e mostra o trabalho que você fez! Não se esqueça de referenciar a figura no texto acima.}

\subsection{Funcionalidades}

O \textit{kw patch\_track} oferece um conjunto de funcionalidades voltadas à automatização e ao gerenciamento das submissões de patches. Todas as interações ocorrem de forma integrada ao fluxo do \textit{kw}, mantendo a compatibilidade com a ferramenta principal de envio.

\subsubsection{Registro e Rastreamento das contribuições}
Durante a submissão dos patches com a ferramenta \textit{kw send\_patch}, uma contribuição pode ser especificada através da flag \textit{--to-contribution=<nome-da-contribuição>}, ou através do terminal interativo que identificará e exibirá as contribuições ainda ativas do usuário para nova submissão além de oferecer a possibilidade de criar uma nova. Nesse momento, ainda de maneira interativa, caso o usuário esteja realizando uma nova contribuição ele tem a possibilidade de identificar um repositório e um mantenedor responsável. Após a submissão, cada novo patch enviado é registrado no banco de dados, juntamente com informações sobre sua versão, título, autor, data de criação, \textit{commit\_hash} e também a sua submissão, caso o patch já tenha sido submetido anteriormente (Caso já exista outro patch no banco de dados com o mesmo título, autor, \textit{commit\_hash} e contribuição), apenas a nova submissão será registrada. Por fim, é registrada uma nova \textit{submission} agrupando as submissões individuais de cada patch enviado na sessão do \textit{kw send-patch} e as relacionando à contribuição.\todo[inline]{Aqui não sei se estou confundindo, mas me fez lembrar de um diagrama que você mostrou na reunião de setembro sobre o caminho e as transições que o patch faria no patch-track em seu ciclo de vida. Seria bem legal ele aparecer aqui.}

Para extrair e salvar as informações dos patches submetidos, a ferramenta se utiliza da técnica de raspagem de dados de dois tipos de arquivos gerados durante a etapa de envio. O primeiro desses arquivos, gerado temporariamente para esse fluxo, é resultado do redirecionamento da saída do comando \textit{git send-email}, utilizado pelo \textit{send-patch} para publicação dos patches. Desse arquivo então o \textit{kw patch-track} extrai grande parte das informações, como o título, email do autor do commit/patch, email do remetente (pode não ser o mesmo usado para criar os commits), os emails dos destinatários, data e horário de submissão e por fim o message-id. Adicionalmente, para ter acesso aos hashes dos commits, avalia-se também os arquivos de patches preliminares, gerados pelo \textit{kw send-patch} para pré-processamento interno. Ainda que parte dos dados extraidos do resultado da submissão estejam disponíveis também no arquivo do patch, o fato de que parte das informações como títulos, autor e destinatários podem ser reescrita durante a submissão somado ao fato de que esses arquivos contém textos adicionais com o conteúdo do patch, poderiam levar a erros de julgamento ou informações imprecisas na hora da extração. 

\subsection{Integração com o mutt}
\subsection{Próximos passos}
Atualmente, o patch-track encontra-se estruturado, com um objetivo claro e possíveis impactos para futuros usuários. Alguns pontos de desenvolvimento que permitiriam que a ferramenta evoluisse para uma versão inicial, incluem:

\subsubsection{Atualização Automática de Status}
Implementar no sistema uma lógica de atualização automática dos status dos patches, baseada em heurísticas inspiradas no fluxo de revisão do \textit{kernel Linux}. Atualizando os estados dos patches e consequentemente das contribuições através de um comando
Os estados possíveis incluem:
\begin{itemize}
\item \textbf{Submetido/Em revisão:} atribuído a patches recém-enviados;
\item \textbf{Revisado:} mantido enquanto há respostas na thread sem substituições;
\item \textbf{Outdated:} aplicado quando uma nova versão substitui a anterior;
\item \textbf{Aprovado:} definido ao detectar respostas contendo marcadores como \textit{Reviewed-by};
\item \textbf{Mergeado:} atribuído quando o hash do commit correspondente é identificado no repositório de destino.
\end{itemize}

\subsubsection{Atualização Manual de Status}
Para permitir mais liberdade ao usuário e eventuais correções de julgamentos da heurística de atualização automática, uma outra implementação útil seria permitir alterações manuais do status do patch, mantendo controle total sobre o histórico.
\subsubsection{Visualização de contribuição/Patches}
Para permitir melhor gestão dos usuários de suas contribuições, permitir uma visualização detalhada tanto das contribuições quanto dos patches submetidos em cada uma
\subsubsection{Renomeação de contribuição}
Permitir que o usuário renomeie suas contribuições caso necessário, uma vez que o nome da contribuição só possui valor para o armazenamento e organização pessoal do usuário
\subsubsection{Realocação de submissões}
Permitir que o usuário realoque as suas submissões de uma contribuição para outra, como a gestão de qual à qual contribuição pertence uma submissão possui sentido apenas para usuário, permitir que ele realoque submissões pode ajudar na correção de eventuais erros de identificação da contribuição durante o envio dos patches
\subsubsection{Definição de repositórios}
Permitir que os usuários determinem à qual repositório será integrada sua contribuição
\subsubsection{Definição de mantenedores}
Permitir que os usuários determinem mantenedores para seus repositórios
\subsubsection{Arquivamento de contribuições}
Permitir que os usuários marquem contribuições como arquivadas, evitando que elas sejam automaticamente incluídas em comandos que interagem com a lista de contribuições

\subsection{Resultados}

\todo[inline]{Aqui, você pode usar o parágrafo abaixo que você comentou como base, focando mais no potencial do que no que foi feito. Novamente, deixando claro que é um WIP, mas que você já avançou bastante e deixou o terreno feito pra outras pessoas finalizarem; aliás, é literalmente o que acontecerá, se você parar para pensar :)}


Com a introdução do \textit{kw patch track}, o processo de contribuição via \textit{kw} deve tornar-se mais organizada e automatizada. A ferramenta deve permitir acompanhar o ciclo de vida de cada patch de forma centralizada, eliminando a necessidade de acompanhamento manual e reduzindo o risco de perda de informações, apresentando maior clareza e rastreabilidade no fluxo de revisões, economia de tempo no acompanhamento de submissões, histórico completo e versionado de cada contribuição e uma base estruturada para análise estatística e integração futura com outras ferramentas, além de um ambiente mais unificado para colaboração no kernel, reduzindo dependências de outras ferramentas como softwares gerenciadores de email. Contudo, apesar dos avanços, dada a necessidade de uma visualização estruturada dos dados, uma das limitações o uso de uma interface de texto via terminal pode melhorar a usabilidade da ferramenta, garantindo melhor adesão da comunidade e melhor impacto na gestão de contribuições e agilidade nessa etapa de desenvolvimento.
