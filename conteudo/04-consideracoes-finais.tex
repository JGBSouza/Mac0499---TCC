\chapter{Considerações Finais}

Esse trabalho apresentou de forma geral, uma análise sobre o kernel Linux, explorando a sua importância e relevância no cenário, suas etapas de desenvolvimento, denominadas por Feitelson como perpetual development até o lançamento de suas versões finais, os “kernels estáveis”. Nesse processo, também foi discutido a relevância do seu desenvolvimento como software livre, o que influencia significativamente a sua segmentação em diversos componentes e o seu modelo de contribuição para permitir a colaboração de uma comunidade de desenvolvedores em sua implementação. Ainda nessa análise, evidencia-se também a complexidade que o sistema adquiriu ao longo dos anos, exigindo uma grande carga de conhecimento técnico e prático antes que possam de fato desenvolver para o sistema.

Nesse contexto, diversas ferramentas surgem de forma a mitigar as dificuldades associadas à esse fluxo de contribuição. Dentre essas ferramentas, o Kernel Workflow se destaca pela busca em oferecer uma interface única e integrada para todas essas dificuldades. Para isso, o kw é construído como um hub modular, integrando funcionalidades locais e externas, oferecendo suporte tanto a tarefas práticas — como a compilação e o deploy de versões do kernel — quanto a processos indiretos, como a submissão de patches, através de comandos de terminal.

Entretanto, compreender e automatizar integralmente o ciclo de contribuição ainda constitui um desafio que o kw busca superar. Dentre as dificuldades ainda não mapeadas, uma das etapas mais críticas é o gerenciamento de patches após a submissão, período em que as contribuições passam por revisões e discussões por parte dos mantenedores e da comunidade de desenvolvedores. Dada a insuficiência das ferramentas de gerenciamento de versão em suprir as necessidades de um software com as dimensões do kernel Linux, hoje, as contribuições para a ferramenta são submetidas através de listas de email, o que representam dificuldades ainda mais significativas para esse processo, como a dependência de ferramentas externas não gerenciáveis, a baixa rastreabilidade e controle das submissões, escalabilidade limitada, além de representar uma ruptura no fluxo de desenvolvimento, que o kw pretende englobar.  

Este trabalho dá continuidade ao desenvolvimento do KWorkflow, integrando-se à linha de evolução de projetos anteriores, como Simplificando o processo de contribuição para o kernel Linux (Neto, 2022) e Integrating the KWorkflow system with the Lore archives (Barros Tadokoro, 2023). Esses trabalhos estabeleceram as bases estruturais do sistema, consolidando o uso de um banco de dados interno, a automatização do envio de patches por e-mail e a integração com os arquivos de discussão oficiais do kernel, sobre as quais o presente estudo se apoia.

A primeira contribuição foca na melhoria do sistema de CRUD do banco de dados, apesar do KWorkflow já contar com funções que permitiam interação com o SQLite3, essas funções não isolavam suficientemente o código das queries, exigindo inserção de trechos SQL (por exemplo WHERE) diretamente nos comandos de seleção; para corrigir isso, os comandos de leitura foram parametrizados para aceitar WHERE parametrizado, ORDER BY por parâmetro e LIMIT,  além da refatoração para permitir que tanto o processo de remoção aceitasse comparações além da igualdade (>, <, >=, <=, !=) e combinações de critérios mais complexas; além disso, foi incorporado o método update\_into para permitir alterações pontuais de atributos em uma entidade essas mudanças melhoraram o fluxo de interação com o banco de dados, viabilizando implementações subsequentes para esse trabalho, como o kw manage contacts e o kw patch track, que dependem de buscas parametrizáveis, ordenação e limites para operar corretamente.

A segunda contribuição, o kw manage-contacts, é uma ferramenta de suporte, permitindo a coordenação de grupos e contatos de contribuidores. Ainda que a ferramenta já possuísse suporte para envio das submissões para mantenedores responsáveis através do comando get\_mantainers, isso não atende alguns grupos não oficiais, como, por exemplo, colegas de trabalho ou outros grupos externos envolvidos com o patch. Como solução, a ferramenta kw\_manage contacts foi desenvolvida, utilizando-se do sistema de banco de dados para armazenar os dados dos contatos e grupos criados pelo usuário bem como das suas relações, denominando de quais grupos cada contato faz parte. Essa ferramenta também se integra diretamente com o sistema de submissões de patches, kw\_manage contacts, permitindo, através dos comandos to-groups e cc-groups, que grupos sejam passados como parâmetro de submissão, garantindo uma submissão muito mais simples e coesa através do terminal de comandos, garantindo corretude nas submissões ao evitar que os contatos precisem ser digitados um a um de forma manual.

Além disso, focando-se de maneira mais específica no problema das listas de email como método de contribuição para o kernel Linux, a terceira implementação desse trabalho, o kw patch-track, foca justamente em oferecer uma ferramenta de gerenciamento local das submissões do usuário. Para isso, a ferramenta se integra também com a ferramenta de envio, kw send-patch, registrando submissões do usuário e utilizando raspagem de dados dos arquivos gerados durante esse processo para extrair as informações necessárias e as registrar no banco de dados. Além disso, através do comando kw patch-track pull, a ferramenta também oferece um sistema de atualização automática do status dos patches, utilizando heurísticas internas para identificar em qual estágio um patch estaria. A integração com o mutt, permite também que, através do terminal, o usuário seja capaz de interagir e visualizar a lista de submissões de suas contribuições, bem como as respostas e revisões de contribuidores e mantenedores da comunidade, podendo até mesmo responde-las. Dessa forma, suprindo a dependência de ferramentas externas para o processo de acompanhamento das revisões, além de fornecer uma solução integrada ao kw, que permite a extração e análise de dados de forma concreta sobre essa etapa para estudos futuros.

A integração dessas ferramentas ao KWorkflow traz impactos significativos tanto para contribuidores experientes quanto para novos participantes do desenvolvimento do kernel Linux. Ao centralizar operações que antes dependiam de múltiplos processos externos — como consultas manuais de listas de e-mail, organização de grupos de contatos e acompanhamento de revisões de patches — o sistema reduz o overhead de contribuintes recém-chegados, simplificando a configuração inicial e a compreensão do fluxo de submissão. O kw manage-contacts garante que grupos de destinatários recorrentes possam ser aplicados automaticamente, evitando erros manuais e agilizando a comunicação, enquanto o kw patch-track oferece rastreabilidade completa das contribuições, permitindo que o usuário visualize o histórico de submissões, respostas e revisões sem recorrer a ferramentas externas.

De forma geral, essas implementações ajudam a consolidar a proposta do KWorkflow de oferecer uma solução unificada e integrada para o ciclo de contribuição ao kernel Linux, promovendo maior previsibilidade, consistência e eficiência. Ao reduzir tarefas repetitivas e centralizar informações críticas, as ferramentas aumentam a produtividade, diminuem a curva de aprendizado e fornecem uma base sólida para automações e análises futuras. Dessa maneira, o trabalho não apenas melhora a experiência do desenvolvedor individual, mas também fortalece o ecossistema colaborativo do kernel, evidenciando a importância de soluções que conectem de forma coerente os diversos elementos do processo de desenvolvimento em um fluxo contínuo e gerenciável.
