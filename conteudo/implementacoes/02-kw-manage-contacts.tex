\section{KW Manage Contacts}

\todo[inline]{Adicionar ``intro da seção''}

\subsection{Objetivos}

No fluxo atual, patches podem ser submetidos através do kw com o uso da ferramenta \textit{kw send-patch}, que recebe como dados a lista de commits que deverão ser enviados e a lista
de usuários, em geral, mantenedores\todo{faltou as listas do subsistema/contexto em questão}, que devem ser notificados da submissão do patch. A partir desse comando, um patch será gerado com as alterações especificadas e, através do \textit{git email},
ferramenta de emails do \textit{github} os usuários receberão um email com as implementações do usuário.\todo[inline]{Na verdade, o kw usa o git send-email por baixo dos panos e ele não tem relação com o GitHub. Aqui vale pincelar como a noção de hub do kw, em que ele agrega a ferramenta que já é consolidada (de certa forma), facilitando e aprimorando o seu uso.}

Ao lidar com e-mails, é sempre comum que hajam grupos de pessoas que sempre serão endereçadas durante a submissão. 
Atualmente, o kw fornece uma funcionalidade, o \textit{get\_maintainers}\todo{get\_maintainers é o script por baixo dos panos; no kw é só kw maintainers} que lista
mantenedores responsáveis pelos sistemas alterados, facilitando a identificação do ou dos responsáveis,
o que, porém, não atende alguns grupos não oficiais, como, por exemplo, colegas de trabalho ou outros grupos
externos envolvidos com o patch. Dessa forma, havia a necessidade de uma ferramenta que pudesse oferecer um sistema de gerenciamento de grupos
de e-mail integrado ao fluxo do kw, garantindo maior praticidade e consistência na comunicação.

Assim, o objetivo principal foi o de criar um sistema que permitisse gerenciar grupos de e-mail de forma centralizada, com armazenamento persistente em
banco de dados\todo{talvez ``[...] armazenamento persistente no banco de dados do kw [...]''} e acesso através de interface em linha de comando (CLI). Permitindo, através disso, que o usuário pudesse cadastrar contatos, organizar esses contatos em
grupos e, posteriormente, incluir automaticamente tais grupos ao enviar patches utilizando o kw send-patch.

\subsection{Arquitetura}

A arquitetura da solução foi planejada de forma modular. O banco de dados é responsável por armazenar contatos individuais (\textit{email\_contact}), os grupos (\textit{email\_group}) e suas associações (\textit{email\_contact\_group}),
enquanto a interface de linha de comando fornece os comandos necessários para manipulação dessas informações. O modelo de dados contempla as entidades contato,
grupo e a relação entre elas, garantindo flexibilidade para gerenciar múltiplos contextos e equipes.\todo[inline]{Seria legal colocar os trechos do kwdb.sql que você implementou. Lembre de omitir o que não for de interesse, pois podemos colocar na íntegra o volume do que você fez nos anexos.}

\subsection{Funcionalidades}
A interação com a ferramenta ocorre exclusivamente pelo terminal, de forma a manter compatibilidade com o fluxo tradicional do kw. 
Foram definidos comandos claros e diretos, permitindo que o usuário visualize grupos existentes, adicione novos contatos, associe-os a
diferentes grupos e utilize esses grupos diretamente no envio de e-mails.

As principais funcionalidades implementadas incluem:\todo{Ao invés de usar os dois pontos, usar o \textbackslash ref\{\} funciona melhor (``[...] principais funcionalidades implementadas estão ilustradas no Programa X.YZ.''}

\begin{programruledcaption}{comandos kw manage contacts}
    \inputminted[breaklines,fontsize=\footnotesize]{bash}{conteudo/implementacoes/codigos/kw_manage_contacts/comandos.bash}
\end{programruledcaption}

\subsubsection{Criação de Grupos}
Essa função recebe o nome de um novo grupo e realiza a validação antes de sua criação. 
A validação consiste em verificar se o nome já não existe no banco de dados, 
se não contém caracteres especiais e se o tamanho é inferior a 50 caracteres. 
Caso todas as condições sejam atendidas, o grupo é criado e persistido no banco com sucesso.\todo{ref ao programa}

\begin{programruledcaption}{create\_email\_group e create\_group}
    \inputminted[breaklines,fontsize=\footnotesize]{bash}{conteudo/implementacoes/codigos/kw_manage_contacts/create_email_group_e_create_group.bash}
\end{programruledcaption}

\subsubsection{Exclusão de Grupos}
Essa função remove um grupo de e-mails específico e todas as suas referências no banco de dados.
Nesse caso, é verificada apenas a existência do grupo, uma vez que, se já foi previamente adicionado, 
deve atender aos critérios de validação. A remoção utiliza a cláusula CASCADE na tabela de associação 
\textit{email\_contact\_group}\todo{se der, mostrar a modelagem}, garantindo que todas as relações existentes para aquele grupo sejam automaticamente 
excluídas. Além disso, os contatos que permanecerem sem associação a grupos também são removidos.\todo{ref ao programa}

\begin{programruledcaption}{remove\_email\_group e remove\_group}
    \inputminted[breaklines,fontsize=\footnotesize]{bash}{conteudo/implementacoes/codigos/kw_manage_contacts/remove_email_group_e_remove_group.bash}
\end{programruledcaption}

\subsubsection{Renomeação de Grupos}
Essa função realiza a renomeação de um grupo de e-mails existente, 
recebendo como parâmetros o nome atual e o novo nome desejado. Antes da atualização,
o novo nome do grupo passa novamente pelo processo de validação, garantindo que não corresponda 
a um grupo já existente no banco de dados e que siga as mesmas regras de restrição aplicadas na 
criação — como não conter caracteres especiais e ter comprimento inferior a 50 caracteres. 
Se as condições forem atendidas, o nome é atualizado com sucesso.\todo{ref ao programa e, se der, mostrar e ref a parte da modelagem}

\begin{programruledcaption}{rename\_email\_group e rename\_group}
    \inputminted[breaklines,fontsize=\footnotesize]{bash}{conteudo/implementacoes/codigos/kw_manage_contacts/rename_email_group_e_rename_group.bash}
\end{programruledcaption}

\subsubsection{Adicionar contatos à Grupos de Email}
Essa função é responsável por associar novos contatos a um grupo de e-mails existente. 
Ela recebe como parâmetros o nome do grupo e uma lista de contatos no formato ``NOME\_CONTATO <EMAIL\_CONTATO>, NOME\_CONTATO <EMAIL\_CONTATO>,
...''. Inicialmente, a função valida a existência do grupo no banco de dados. Em seguida, divide a lista de contatos e executa verificações
individuais para cada um, como a validade do endereço de e-mail e a presença de um nome associado. Após essas validações, os contatos são 
inseridos no banco de dados e vinculados ao grupo correspondente.\todo{ref ao programa e, se der, mostrar e ref a parte da modelagem}

\begin{programruledcaption}{add\_email\_contacts e add\_contact\_group}
    \inputminted[breaklines,fontsize=\footnotesize]{bash}{conteudo/implementacoes/codigos/kw_manage_contacts/add_email_contacts_e_add_contact_group.bash}
\end{programruledcaption}

\subsubsection{Exibir grupos}
A função show\_email\_groups\todo{ref ao programa e colocar em itálico como você estava convencionando} é utilizada para exibir informações sobre os grupos de e-mail cadastrados. 
Caso receba como parâmetro o nome de um grupo específico, a função valida sua existência e, 
em seguida, mostra os contatos associados a esse grupo. Por outro lado, quando nenhum parâmetro 
é informado, são exibidas informações gerais sobre todos os grupos existentes no banco de dados. 

\begin{programruledcaption}{show email groups, print\_groups\_infos e print\_contacts\_infos}
    \inputminted[breaklines,fontsize=\footnotesize]{bash}{conteudo/implementacoes/codigos/kw_manage_contacts/show_email_groups.bash}
\end{programruledcaption}

\begin{figure}[H]
  \centering
    \includegraphics[width=1.2\textwidth]{exemplo\_group\_show}
    \caption{Exemplo do comando ``group\_show''  sem um grupo especificado.\label{fig:exemplo_group_show:a}}
\end{figure}

\begin{figure}[H]
  \centering
    \includegraphics[width=1.2\textwidth]{exemplo_group_show_especifico}
    \caption{Exemplo do comando ``group\_show'' para um grupo específico.\label{fig:exemplo_group_show_especifico:a}}
\end{figure}

\subsection{Enviar patches para grupos}

Dois novos comandos, \textit{--to-groups} e \textit{--cc-groups} foram adicionados à ferramenta \textit{send-patch} para permitir o envio de patches diretamente para grupos.\todo{ref ao programa}
Esses comandos recebem listas com os nomes dos grupos separados por vírgulas, busca os contatos associados no banco de dados e adiciona seus emails como destinatários.

\begin{programruledcaption}{opções do comando --send do kw send-patch contendo o to-groups e o cc-groups}
    \inputminted[breaklines,fontsize=\footnotesize]{bash}{conteudo/implementacoes/codigos/kw_manage_contacts/kw_send_patch_options.bash}
\end{programruledcaption}

\begin{programruledcaption}{Função send\_patch\_main com métodos --to-groups e cc-groups}
    \inputminted[breaklines,fontsize=\footnotesize]{bash}{conteudo/implementacoes/codigos/kw_manage_contacts/kw_send_patch.bash}
\end{programruledcaption}

\subsection{Resultados}
Entre os benefícios da abordagem adotada estão a maior praticidade no gerenciamento de destinatários, a redução de erros manuais na inclusão
de e-mails e a possibilidade de reutilização de grupos em diferentes contextos. Isso se traduz em um processo mais ágil e confiável no envio de patches.

Apesar dos avanços alcançados, algumas limitações ainda podem ser apontadas. 
A ferramenta oferece suporte apenas via CLI, não possuindo interface gráfica que poderia ser desejável, principalmente para visualizar
informações dos grupos e contatos de uma maneira mais organizada.\todo[inline]{Acho que, ao invés de falar de GUI, falar de formas de se importar ou até exportar os grupos. Sei que existem aqueles arquivos de contact books, mas veja se procede essa ideia; de toda forma, GUI não é muito a linha do kw. Se estiver se referindo a uma TUI na linha do patch-hub aí faz sentido, mas seja explícito, porque interface gráfica é associado a GUI.}
