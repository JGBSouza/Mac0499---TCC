\section{KW Manage Contacts}

No fluxo atual, a submissão de \textit{patches} pode ser realizada por meio da ferramenta \textit{kw send-patch}. Este comando recebe como parâmetros a lista de \textit{commits} a serem enviados, os usuários responsáveis — geralmente mantenedores — e as listas de discussão do subsistema pertinente que devem ser notificadas da contribuição. A partir desses dados, o \textit{kw} gera as modificações necessárias e utiliza internamente o \textit{git send-email} para a transmissão das mensagens aos destinatários. Essa abordagem reflete a função do \textit{kw} como um \textit{hub}, que agrega e simplifica o uso de ferramentas já consolidadas na comunidade, automatizando etapas manuais e aprimorando o fluxo de submissão do desenvolvedor.

Ao lidar com e-mails, é comum que existam grupos de destinatários recorrentes durante a submissão de \textit{patches}. Atualmente, o \textit{kw} disponibiliza a funcionalidade \textit{kw maintainers} que, ao utilizar internamente o \textit{script get\_maintainers.pl} do kernel, lista os mantenedores responsáveis pelos subsistemas alterados. Embora essa ferramenta facilite a identificação dos responsáveis oficiais, ela não contempla grupos não oficiais ou externos, como equipes de trabalho e colaboradores de projetos específicos. Diante disso, surgiu a necessidade de uma ferramenta integrada ao fluxo do \textit{kw} para o gerenciamento de grupos de e-mail, visando garantir maior praticidade e consistência na comunicação do desenvolvedor.

\subsection{Objetivos}

Assim, o objetivo principal foi o de criar um sistema que permitisse gerenciar grupos de e-mail de forma centralizada, com armazenamento persistente no banco de dados do kw e acesso através de interface em linha de comando (CLI). Permitindo, através disso, que o usuário pudesse cadastrar contatos, organizar esses contatos em
grupos e, posteriormente, incluir automaticamente tais grupos ao enviar patches utilizando o kw send-patch.

\subsection{Arquitetura}

A arquitetura da solução foi planejada de forma modular. O banco de dados é responsável por armazenar contatos individuais (\textit{email\_contact}), os grupos (\textit{email\_group}) e suas associações (\textit{email\_contact\_group}),
enquanto a interface de linha de comando fornece os comandos necessários para manipulação dessas informações. O modelo de dados contempla as entidades contato,
grupo e a relação entre elas, garantindo flexibilidade para gerenciar múltiplos contextos e equipes. (Figura~\ref{fig:kw_manage_contacts_erd})

\begin{figure}[H]
    \centering
    \includegraphics[width=0.8\textwidth]{kw_manage_contacts_erd}
    \caption{Diagrama Entidade-Relacionamento do Kw manage-contacts}
    \label{fig:kw_manage_contacts_erd}
\end{figure}

\begin{programruledcaption}{modelagem sql Kw manage-contacts \label{prog:kw_manage_contacts_sql}}
    \inputminted[breaklines,fontsize=\footnotesize]{sql}{conteudo/implementacoes/codigos/kw_manage_contacts/kw_manage_contacts_db.sql}
\end{programruledcaption}

\subsection{Funcionalidades}
A interação com a ferramenta ocorre exclusivamente pelo terminal, de forma a manter compatibilidade com o fluxo tradicional do kw. 
Foram definidos comandos claros e diretos, permitindo que o usuário visualize grupos existentes, adicione novos contatos, associe-os a
diferentes grupos e utilize esses grupos diretamente no envio de e-mails.

As principais funcionalidades implementadas são referenciadas no Programa~\ref{prog:kw_manage_contacts_comandos}

\begin{programruledcaption}{comandos kw manage contacts. \label{prog:kw_manage_contacts_comandos}}
    \inputminted[breaklines,fontsize=\footnotesize]{bash}{conteudo/implementacoes/codigos/kw_manage_contacts/comandos.bash}
\end{programruledcaption}

\subsubsection{Criação de Grupos}
Essa função recebe o nome de um novo grupo e realiza a validação antes de sua criação. 
A validação consiste em verificar se o nome já não existe no banco de dados, 
se não contém caracteres especiais e se o tamanho é inferior a 50 caracteres. 
Caso todas as condições sejam atendidas, o grupo é criado e persistido no banco com sucesso (Programa~\ref{prog:kw_mc_create_email_group}).

\begin{programruledcaption}{create\_email\_group e create\_group. \label{prog:kw_mc_create_email_group}}
    \inputminted[breaklines,fontsize=\footnotesize]{bash}{conteudo/implementacoes/codigos/kw_manage_contacts/create_email_group_e_create_group.bash}
\end{programruledcaption}

\subsubsection{Exclusão de Grupos}

Essa função remove um grupo de e-mails específico e todas as suas referências no banco de dados.
Nesse caso, é verificada apenas a existência do grupo, uma vez que, se já foi previamente adicionado, 
deve atender aos critérios de validação. A remoção utiliza a cláusula CASCADE na tabela de associação (Programa~\ref{prg:kw_manage_contacts_sql}),
garantindo que todas as relações existentes para aquele grupo sejam automaticamente 
excluídas. Além disso, os contatos que permanecerem sem associação a grupos também são removidos (Programa~\ref{prog:kw_mc_remove_email_group}).

\begin{programruledcaption}{remove\_email\_group e remove\_group. \label{prog:kw_mc_remove_email_group}}
    \inputminted[breaklines,fontsize=\footnotesize]{bash}{conteudo/implementacoes/codigos/kw_manage_contacts/remove_email_group_e_remove_group.bash}
\end{programruledcaption}

\subsubsection{Renomeação de Grupos}
Essa função realiza a renomeação de um grupo de e-mails existente, 
recebendo como parâmetros o nome atual e o novo nome desejado. Antes da atualização,
o novo nome do grupo passa novamente pelo processo de validação, garantindo que não corresponda 
a um grupo já existente no banco de dados e que siga as mesmas regras de restrição aplicadas na 
criação — como não conter caracteres especiais e ter comprimento inferior a 50 caracteres. 
Se as condições forem atendidas, o nome é atualizado com sucesso (Programa~\cite{prog:kw_mc_rename_email_group})

\begin{programruledcaption}{rename\_email\_group e rename\_group. \label{prog:kw_mc_rename_email_group}}
    \inputminted[breaklines,fontsize=\footnotesize]{bash}{conteudo/implementacoes/codigos/kw_manage_contacts/rename_email_group_e_rename_group.bash}
\end{programruledcaption}

\subsubsection{Adicionar contatos à Grupos de Email}
Essa função é responsável por associar novos contatos a um grupo de e-mails existente. 
Ela recebe como parâmetros o nome do grupo e uma lista de contatos no formato ``NOME\_CONTATO <EMAIL\_CONTATO>, NOME\_CONTATO <EMAIL\_CONTATO>,
...''. Inicialmente, a função valida a existência do grupo no banco de dados. Em seguida, divide a lista de contatos e executa verificações
individuais para cada um, como a validade do endereço de e-mail e a presença de um nome associado. Após essas validações, os contatos são 
inseridos no banco de dados e vinculados ao grupo correspondente (Programa~\ref{prog:kw_mc_rename_email_group}).

\begin{programruledcaption}{add\_email\_contacts e add\_contact\_group \label{prog:kw_mc_rename_email_group}}
    \inputminted[breaklines,fontsize=\footnotesize]{bash}{conteudo/implementacoes/codigos/kw_manage_contacts/add_email_contacts_e_add_contact_group.bash}
\end{programruledcaption}

\subsubsection{Exibir grupos}
A função \textit{show\_email\_groups}, Programa~\ref{prog:show_email_groups}, é utilizada para exibir informações sobre os grupos de e-mail cadastrados. 
Caso receba como parâmetro o nome de um grupo específico, a função valida sua existência e, 
em seguida, mostra os contatos associados a esse grupo. Por outro lado, quando nenhum parâmetro 
é informado, são exibidas informações gerais sobre todos os grupos existentes no banco de dados. 

\begin{programruledcaption}{show email groups, print\_groups\_infos e print\_contacts\_infos. \label{prog:show_email_groups}}
    \inputminted[breaklines,fontsize=\footnotesize]{bash}{conteudo/implementacoes/codigos/kw_manage_contacts/show_email_groups.bash}
\end{programruledcaption}

\begin{figure}[H]
  \centering
    \includegraphics[width=1.2\textwidth]{exemplo\_group\_show}
    \caption{Exemplo do comando ``group\_show''  sem um grupo especificado. \label{fig:exemplo_group_show}}
\end{figure}

\begin{figure}[H]
  \centering
    \includegraphics[width=1.2\textwidth]{exemplo_group_show_especifico}
    \caption{Exemplo do comando ``group\_show'' para um grupo específico. \label{fig:exemplo_group_show_especifico}}
\end{figure}

\subsection{Enviar patches para grupos}

Dois novos comandos, \textit{--to-groups} e \textit{--cc-groups} foram adicionados à ferramenta \textit{send-patch} para permitir o envio de patches diretamente para grupos.
Esses comandos recebem listas com os nomes dos grupos separados por vírgulas, busca os contatos associados no banco de dados e adiciona seus emails como destinatários (Programa~\cite{send_patch_main_com_to_cc_groups}).

\begin{programruledcaption}{opções do comando --send do kw send-patch contendo o to-groups e o cc-groups. \label{prog:opcoes-send-patch}}
    \inputminted[breaklines,fontsize=\footnotesize]{bash}{conteudo/implementacoes/codigos/kw_manage_contacts/kw_send_patch_options.bash}
\end{programruledcaption}

\begin{programruledcaption}{Função send\_patch\_main com métodos --to-groups e cc-groups. \label{prog:send_patch_main_com_to_cc_groups}}
    \inputminted[breaklines,fontsize=\footnotesize]{bash}{conteudo/implementacoes/codigos/kw_manage_contacts/kw_send_patch.bash}
\end{programruledcaption}

\subsection{Resultados}
Entre os benefícios da abordagem adotada estão a maior praticidade no gerenciamento de destinatários, a redução de erros manuais na inclusão
de e-mails e a possibilidade de reutilização de grupos em diferentes contextos. Isso se traduz em um processo mais ágil e confiável no envio de patches.

Apesar dos avanços alcançados, algumas limitações ainda podem ser apontadas. A ferramenta oferece suporte apenas via CLI, não possuindo interface para o usuário via terminal, que poderia ser desejável, principalmente, para visualizar informações dos grupos e contatos de uma maneira mais organizada.