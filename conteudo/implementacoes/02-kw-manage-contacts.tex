\section{KW Manage Contacts}

No fluxo atual, a submissão de \textit{patches} pode ser realizada por meio da ferramenta \textit{kw send-patch}. Este comando recebe como parâmetros a lista de \textit{commits} a serem enviados, os usuários responsáveis — geralmente mantenedores — e as listas de discussão do subsistema pertinente que devem ser notificadas da contribuição. A partir desses dados, o \textit{kw} gera as modificações necessárias e utiliza internamente o \textit{git send-email} para a transmissão das mensagens aos destinatários. Essa abordagem reflete a função do \textit{kw} como um \textit{hub}, que agrega e simplifica o uso de ferramentas já consolidadas na comunidade, automatizando etapas manuais e aprimorando o fluxo de submissão do desenvolvedor.

Ao lidar com e-mails, é comum que existam grupos de destinatários recorrentes durante a submissão de \textit{patches}. Atualmente, o \textit{kw} disponibiliza a funcionalidade \textit{kw maintainers} que, ao utilizar internamente o \textit{script get\_maintainers.pl} do kernel, lista os mantenedores responsáveis pelos subsistemas alterados. Embora essa ferramenta facilite a identificação dos responsáveis oficiais, ela não contempla grupos não oficiais ou externos, como equipes de trabalho e colaboradores de projetos específicos. Diante disso, surgiu a necessidade de uma ferramenta integrada ao fluxo do \textit{kw} para o gerenciamento de grupos de e-mail, visando garantir maior praticidade e consistência na comunicação do desenvolvedor.

\subsection{Objetivos}

Assim, o objetivo principal foi o de criar um sistema que permitisse gerenciar grupos de e-mail de forma centralizada, com armazenamento persistente no banco de dados do kw e acesso através de interface em linha de comando (CLI). Permitindo, através disso, que o usuário pudesse cadastrar contatos, organizar esses contatos em
grupos e, posteriormente, incluir automaticamente tais grupos ao enviar patches utilizando o kw send-patch.

\subsection{Arquitetura}

A arquitetura da solução foi planejada de forma modular. O banco de dados é responsável por armazenar contatos individuais (\textit{email\_contact}), os grupos (\textit{email\_group}) e suas associações (\textit{email\_contact\_group}),
enquanto a interface de linha de comando fornece os comandos necessários para manipulação dessas informações. O modelo de dados contempla as entidades contato,
grupo e a relação entre elas, garantindo flexibilidade para gerenciar múltiplos contextos e equipes (Figura~\ref{fig:kw_manage_contacts_erd}).

\begin{figure}[!htbp]
    \centering
    \includegraphics[width=0.8\textwidth]{kw_manage_contacts_erd}
    \caption{Diagrama Entidade-Relacionamento do kw manage-contacts}
    \label{fig:kw_manage_contacts_erd}
\end{figure}

\begin{programruledcaption}{modelagem sql kw manage-contacts \label{prog:kw_manage_contacts_sql}}
    \inputminted[breaklines,fontsize=\footnotesize]{sql}{conteudo/implementacoes/codigos/kw_manage_contacts/kw_manage_contacts_db.sql}
\end{programruledcaption}

\subsection{Funcionalidades}
A interação com a ferramenta ocorre exclusivamente pelo terminal, de forma a manter compatibilidade com o fluxo tradicional do kw. 
Foram definidos comandos claros e diretos, permitindo que o usuário visualize grupos existentes, adicione novos contatos, associe-os a
diferentes grupos e utilize esses grupos diretamente no envio de e-mails.

As principais funcionalidades implementadas são referenciadas no Programa~\ref{prog:kw_manage_contacts_comandos}.

\begin{programruledcaption}{comandos kw manage contacts. \label{prog:kw_manage_contacts_comandos}}
    \inputminted[breaklines,fontsize=\footnotesize]{bash}{conteudo/implementacoes/codigos/kw_manage_contacts/comandos.bash}
\end{programruledcaption}

\subsubsection{Criação de Grupos}

A funcionalidade de criação de grupos, executada por meio do comando \texttt{kw manage-contacts group-create}, permite a organização estruturada de contatos para facilitar as submissões. O comando recebe o nome do novo grupo, que é submetido a algumas de validações de integridade antes de sua persistência no banco de dados. Essas verificações asseguram que o identificador proposto seja único, não contenha caracteres especiais e respeite o limite de 50 caracteres. Uma vez atendidos os requisitos, o sistema realiza a inserção do registro (Programa~\ref{prog:kw_mc_create_email_group}).

\begin{programruledcaption}{create\_email\_group e create\_group. \label{prog:kw_mc_create_email_group}}
    \inputminted[breaklines,fontsize=\footnotesize]{bash}{conteudo/implementacoes/codigos/kw_manage_contacts/create_email_group_e_create_group.bash}
\end{programruledcaption}

\subsubsection{Exclusão de Grupos}

A funcionalidade de exclusão, invocada pelo comando \texttt{kw manage-contacts group-remove}, permite a remoção definitiva de uma categoria de contatos e de todas as suas referências no sistema. A operação exige a validação da existência prévia do grupo no banco de dados e utiliza a cláusula \textit{CASCADE} na tabela de associação para garantir a consistência dos dados (Programa~\ref{prog:kw_manage_contacts_sql}). Este mecanismo extingue automaticamente todos os vínculos do grupo, enquanto uma rotina adicional remove contatos que permaneçam sem qualquer outra associação (Programa~\ref{prog:kw_mc_remove_email_group}).

\begin{programruledcaption}{remove\_email\_group e remove\_group. \label{prog:kw_mc_remove_email_group}}
    \inputminted[breaklines,fontsize=\footnotesize]{bash}{conteudo/implementacoes/codigos/kw_manage_contacts/remove_email_group_e_remove_group.bash}
\end{programruledcaption}

\subsubsection{Renomeação de Grupos}

A funcionalidade de renomeação, acessada através do comando \texttt{kw manage-contacts group-rename}, permite a alteração de identificadores existentes sem a necessidade de excluir e recriar registros. A operação recebe como parâmetros o nome atual e o novo rótulo, submetendo este último ao mesmo processo de validação do ciclo de criação (limite de caracteres e ausência de símbolos especiais). Essa revalidação assegura que a consistência da base de dados seja preservada, mantendo íntegras as associações de contatos vinculadas ao grupo (Programa~\ref{prog:kw_mc_rename_email_group}).

\begin{programruledcaption}{rename\_email\_group e rename\_group. \label{prog:kw_mc_rename_email_group}}
    \inputminted[breaklines,fontsize=\footnotesize]{bash}{conteudo/implementacoes/codigos/kw_manage_contacts/rename_email_group_e_rename_group.bash}
\end{programruledcaption}

\subsubsection{Adicionar contatos à Grupos de Email}

A funcionalidade de associação de contatos, executada pelo comando \texttt{kw group-add-contact}, permite o gerenciamento e a expansão de grupos por meio da inserção em lote. A operação recebe o nome do grupo alvo e uma lista estruturada no formato \texttt{NOME <EMAIL>}. O sistema valida a existência do grupo, segmenta a entrada e executa verificações de integridade sintática em cada endereço de e-mail. Uma vez validados, os contatos são persistidos e vinculados ao grupo correspondente (Programa~\ref{prog:kw_add_email_contact}).

\begin{programruledcaption}{add\_email\_contacts e add\_contact\_group \label{prog:kw_add_email_contact}}
    \inputminted[breaklines,fontsize=\footnotesize]{bash}{conteudo/implementacoes/codigos/kw_manage_contacts/add_email_contacts_e_add_contact_group.bash}
\end{programruledcaption}

\subsubsection{Exibir grupos}

A funcionalidade de consulta, operada pelo comando \texttt{kw manage-contacts group-show}, permite visualizar as informações armazenadas no banco de dados de duas maneiras. Quando um identificador de grupo é fornecido como parâmetro, o sistema valida sua existência e lista todos os contatos vinculados (Figura~\ref{fig:exemplo_group_show_especifico}). Caso o comando seja invocado sem parâmetros, o sistema apresenta um resumo de todos os grupos cadastrados (Figura~\ref{fig:exemplo_group_show}). A lógica de processamento e formatação da saída de dados detalhada pode ser observada no Programa~\ref{prog:show_email_groups}.

\begin{programruledcaption}{show email groups, print\_groups\_infos e print\_contacts\_infos. \label{prog:show_email_groups}}
    \inputminted[breaklines,fontsize=\footnotesize]{bash}{conteudo/implementacoes/codigos/kw_manage_contacts/show_email_groups.bash}
\end{programruledcaption}

\begin{figure}[!htbp]
  \centering
    \includegraphics[width=1.2\textwidth]{exemplo_group_show_especifico}
    \caption{Exemplo do comando ``group\_show'' para um grupo específico. \label{fig:exemplo_group_show_especifico}}
\end{figure}

\begin{figure}[!htbp]
  \centering
    \includegraphics[width=1.2\textwidth]{exemplo\_group\_show}
    \caption{Exemplo do comando ``group\_show''  sem um grupo especificado. \label{fig:exemplo_group_show}}
\end{figure}
\subsection{Enviar patches para grupos}

A integração de grupos de e-mail ao fluxo de submissões é viabilizada pelos novos parâmetros \texttt{--to-groups} e \texttt{--cc-groups}, incorporadas a feature \textit{kw send-patch}. Essa funcionalidade permite ao usuário enviar patches a grupos de contatos pré-definidos, eliminando a necessidade de inserção manual de múltiplos endereços, principalmente em submissões recorrentes.

Para que isso seja possível, o \textit{send-patch} recebe o nome dos grupos em listas contendo identificadores de grupos separados por vírgulas. Internamente, o sistema processa essa entrada realizando consultas ao banco de dados para realizar a consulta dos nomes de grupos e encontrar os endereços de e-mail válidos, que são então injetados nos campos de destinatário (\textit{To}) ou de cópia (\textit{Cc}) da mensagem (Programa~\ref{prog:send_patch_main_com_to_cc_groups}). Os novos parâmetros podem ser encontrados em: Programas~\ref{prog:opcoes-send-patch}.

\begin{programruledcaption}{opções do comando --send do kw send-patch contendo o to-groups e o cc-groups. \label{prog:opcoes-send-patch}}
    \inputminted[breaklines,fontsize=\footnotesize]{bash}{conteudo/implementacoes/codigos/kw_manage_contacts/kw_send_patch_options.bash}
\end{programruledcaption}

\begin{programruledcaption}{Função send\_patch\_main com métodos --to-groups e cc-groups. \label{prog:send_patch_main_com_to_cc_groups}}
    \inputminted[breaklines,fontsize=\footnotesize]{bash}{conteudo/implementacoes/codigos/kw_manage_contacts/kw_send_patch.bash}
\end{programruledcaption}

\subsection{Resultados}
Entre os benefícios da abordagem adotada estão a maior praticidade no gerenciamento de destinatários, a redução de erros manuais na inclusão
de e-mails e a possibilidade de reutilização de grupos em diferentes contextos. Isso se traduz em um processo mais ágil e confiável no envio de patches.

Apesar dos avanços alcançados, algumas limitações ainda podem ser apontadas. A ferramenta oferece suporte apenas via CLI, não possuindo interface para o usuário via terminal, que poderia ser desejável, principalmente, para visualizar informações dos grupos e contatos de uma maneira mais organizada.