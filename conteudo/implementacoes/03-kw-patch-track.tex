\section{KW Patch track}

Atualmente, embora seja possível submeter \textit{patches} através do \textit{kw}, ainda não existe um mecanismo eficaz para acompanhar e gerenciar o ciclo de vida dessas submissões. Conforme novas versões de um mesmo \textit{patch} são enviadas e as revisões se acumulam, torna-se cada vez mais difícil manter o controle sobre o histórico, as respostas recebidas e o estado atual de cada alteração. Esta lacuna é significativa, pois obriga o desenvolvedor a realizar esse acompanhamento de forma externa e manual, o que fragmenta o fluxo de trabalho e aumenta a carga cognitiva. Diante dessa limitação, surgiu a necessidade de uma funcionalidade capaz de registrar, rastrear e atualizar automaticamente o status dos \textit{patches} submetidos, centralizando o gerenciamento do ciclo de vida diretamente no ecossistema do \textit{kw}.

\subsection{Objetivos}
Assim, o objetivo principal do \textit{Patch Track} é permitir que o usuário acompanhe de forma automatizada o progresso de suas contribuições, desde o envio inicial até a integração no repositório, reduzindo o esforço manual e promovendo maior clareza sobre o processo de revisão.

Além disso, o sistema busca oferecer uma base sólida para extensões futuras, como integração com repositórios oficiais e coleta de métricas sobre o fluxo de contribuição, incluindo tempo médio de resposta, aprovação e integração de patches. Essa capacidade de extração de dados é fundamental para consolidar o \textit{kw} como uma ferramenta de suporte à pesquisa científica, permitindo o estudo empírico e sistemático do modelo de desenvolvimento do kernel Linux e o entendimento aprofundado de suas dinâmicas de colaboração em larga escala.

\subsection{Arquitetura}

A arquitetura do \textit{Patch Track} foi projetada de forma modular e baseada em um modelo relacional de entidades interligadas. Todas as informações são armazenadas em banco de dados, garantindo rastreabilidade e consistência das submissões.

A entidade central, \textit{patch}, armazena informações como o autor, o \textit{message-id} da submissão, a versão e o status atual do patch. O campo \textit{outdated} indica quando uma versão mais recente substitui outra, preservando o histórico completo das alterações.

A entidade \textit{contribution} agrupa logicamente diferentes versões de um mesmo trabalho, mantendo informações sobre a data da última interação e o repositório de destino. Esse repositório é representado pela entidade \textit{repository}, que contém dados como nome, URL e \textit{branch} associada na qual a contribuição deve ser integrada assim que aprovada, além dos mantenedores vinculados à esse repositório, permitindo identificar revisores e correlacionar respostas relevantes nas threads de e-mail.

O rastreamento dos envios é realizado pela tabela \textit{patch\_submission}, que registra o identificador da mensagem, o remetente e o vínculo entre cada envio e o patch correspondente. O sistema também oferece suporte a \textit{tags}, utilizadas como marcadores semânticos para facilitar a filtragem, a categorização e a exibição das informações.

Essa estrutura de dados estabelece uma base robusta para o controle do ciclo de vida dos patches e possibilita futuras expansões, como integração com serviços externos de revisão e automação de métricas analíticas (Figura~\ref{fig:kw_patch_track_erd}).

\begin{figure}[H]
    \centering
    \includegraphics[width=0.8\textwidth]{kw_patch_track_erd}
    \caption{Diagrama Entidade-Relacionamento do Kw patch\_track}
    \label{fig:kw_patch_track_erd}
\end{figure}

\subsection{Funcionalidades}

O \textit{kw patch\_track} oferece um conjunto de funcionalidades voltadas à automatização e ao gerenciamento das submissões de patches. Todas as interações ocorrem de forma integrada ao fluxo do \textit{kw}, mantendo a compatibilidade com a ferramenta principal de envio.

\subsubsection{Registro e Rastreamento das contribuições}
Durante a submissão dos patches com a ferramenta \textit{kw send\_patch}, uma contribuição pode ser especificada através da flag \textit{--to-contribution=<nome-da-contribuição>}, ou através do terminal interativo que identificará e exibirá as contribuições ainda ativas do usuário para nova submissão além de oferecer a possibilidade de criar uma nova. Nesse momento, ainda de maneira interativa, caso o usuário esteja realizando uma nova contribuição ele tem a possibilidade de identificar um repositório e um mantenedor responsável. Após a submissão, cada novo patch enviado é registrado no banco de dados, juntamente com informações sobre sua versão, título, autor, data de criação, \textit{commit\_hash} e também a sua submissão, caso o patch já tenha sido submetido anteriormente (Caso já exista outro patch no banco de dados com o mesmo título, autor, \textit{commit\_hash} e contribuição), apenas a nova submissão será registrada. Por fim, é registrada uma nova \textit{submission} agrupando as submissões individuais de cada patch enviado na sessão do \textit{kw send-patch} e as relacionando à contribuição.

Para extrair e salvar as informações dos patches submetidos, a ferramenta se utiliza da técnica de raspagem de dados de dois tipos de arquivos gerados durante a etapa de envio. O primeiro desses arquivos, gerado temporariamente para esse fluxo, é resultado do redirecionamento da saída do comando \textit{git send-email}, utilizado pelo \textit{send-patch} para publicação dos patches. Desse arquivo então o \textit{kw patch-track} extrai grande parte das informações, como o título, email do autor do commit/patch, email do remetente (pode não ser o mesmo usado para criar os commits), os emails dos destinatários, data e horário de submissão e por fim o message-id. Adicionalmente, para ter acesso aos hashes dos commits, avalia-se também os arquivos de patches preliminares, gerados pelo \textit{kw send-patch} para pré-processamento interno. Ainda que parte dos dados extraidos do resultado da submissão estejam disponíveis também no arquivo do patch, o fato de que parte das informações como títulos, autor e destinatários podem ser reescrita durante a submissão somado ao fato de que esses arquivos contém textos adicionais com o conteúdo do patch, poderiam levar a erros de julgamento ou informações imprecisas na hora da extração. 

\subsubsection{Integração com o mutt}
Permitir integração com o cliente de email via terminal \textit{Mutt}, permitindo analisar caixas de email do usuário, monitorando submissões e eventuais respostas, além de permitir a visualização dos emails diretamente via terminal. 


\subsubsection{Atualização Automática de Status}
Implementar no sistema uma lógica de atualização automática dos status dos patches, baseada em heurísticas inspiradas no fluxo de revisão do \textit{kernel Linux}. Atualizando os estados dos patches e consequentemente das contribuições através de um comando

Os estados possíveis incluem:
\begin{itemize}
\item \textbf{Submetido/Em revisão:} atribuído a patches recém-enviados;
\item \textbf{Revisado:} mantido enquanto há respostas na thread sem substituições;
\item \textbf{Outdated:} aplicado quando uma nova versão substitui a anterior;
\item \textbf{Aprovado:} definido ao detectar respostas contendo marcadores como \textit{Reviewed-by};
\item \textbf{Mergeado:} atribuído quando o hash do commit correspondente é identificado no repositório de destino.
\end{itemize}

\subsubsection{Atualização Manual de Status}
Para permitir mais liberdade ao usuário e eventuais correções de julgamentos da heurística de atualização automática, uma outra implementação útil seria permitir alterações manuais do status do patch, mantendo controle total sobre o histórico.

\subsubsection{Visualização de contribuição/Patches}
Para permitir melhor gestão dos usuários de suas contribuições, permitir uma visualização detalhada tanto das contribuições quanto dos patches submetidos em cada uma

\subsubsection{Definição de repositórios}
Permitir que os usuários determinem à qual repositório será integrada sua contribuição

\subsubsection{Definição de mantenedores}
Permitir que os usuários determinem mantenedores para seus repositórios

\subsection{Próximos passos}
Atualmente, o patch-track encontra-se estruturado, com um objetivo claro e possíveis impactos para futuros usuários. Alguns pontos de desenvolvimento que permitiriam que a ferramenta evoluisse para uma versão inicial, incluem:

\subsubsection{Renomeação de contribuição}
Permitir que o usuário renomeie suas contribuições caso necessário, uma vez que o nome da contribuição só possui valor para o armazenamento e organização pessoal do usuário

\subsubsection{Realocação de submissões}
Permitir que o usuário realoque as suas submissões de uma contribuição para outra, como a gestão de qual à qual contribuição pertence uma submissão possui sentido apenas para usuário, permitir que ele realoque submissões pode ajudar na correção de eventuais erros de identificação da contribuição durante o envio dos patches

\subsection{Resultados}
Com a introdução do \textit{kw patch track}, o processo de contribuição via \textit{kw} deve tornar-se mais organizado e automatizado. A ferramenta deve permitir acompanhar o ciclo de vida de cada patch de forma centralizada, eliminando a necessidade de acompanhamento manual e reduzindo o risco de perda de informações, apresentando maior clareza e rastreabilidade no fluxo de revisões, economia de tempo no acompanhamento de submissões, histórico completo e versionado de cada contribuição, uma base estruturada para análise estatística e integração futura com outras ferramentas, além de um ambiente mais unificado para colaboração no kernel, reduzindo dependências de outras ferramentas, como softwares gerenciadores de email.